\chapter{Instrucciones de gretl}
\label{cmdref}

\section{Notaci�n}
\label{cmd-intro}

Las instrucciones definidas en esta secci�n pueden ejecutarse en el
programa cliente de l�nea de instrucciones. Tambi�n pueden incluirse
en un archivo o ``lote de instrucciones'' (script) y as� ejecutarse en
el GUI, o teclearse mediante el modo consola de este �ltimo. En la
mayor�a de los casos la sintaxis que se menciona es tambi�n aplicable
para rellenar una l�nea en el correspondiente cuadro de di�logo del
GUI (v�ase tambi�n la ayuda \emph{en l�nea} de \app{gretl}), excepto
que \emph{no} es preciso teclear la palabra inicial de la instrucci�n
--- est� impl�cita por el contexto. Una diferencia adicional es que no
se puede insertar la marca \cmd{-o} para las instrucciones de
regresi�n en los cuadros de di�logo del GUI: hay una opci�n de men�
para mostrar la matriz formada por las varianzas y covarianzas de los
coeficientes (que es el efecto de \cmd{-o} en las instrucciones de las
regresiones).

A lo largo de este cap�tulo se utilizan las siguientes convenciones:


\begin{itemize}
\item La fuente \texttt{typewriter} se utiliza en lo que tecleamos
  directamente, y tambi�n para los nombres internos de las variables.

\item Los t�rminos en \textsl{cursiva} son marcadores de ubicaci�n: es
  posible sustituirlos por algo m�s espec�fico, por ejemplo, se puede
  escribir \texttt{renta} en lugar del gen�rico \textsl{varx}.

\item \texttt{[ -o ]} significa que la marca \cmd{-o} es opcional:
  puede ser a�adida o no (pero en todo caso sin los par�ntesis).

\item La frase ``instrucci�n de estimaci�n'' puede significar
  cualquiera de las siguientes \cmd{ols}, \cmd{hilu}, \cmd{corc},
  \cmd{ar}, \cmd{arch}, \cmd{hsk}, \cmd{tsls}, \cmd{wls}, \cmd{hccm},
  \cmd{add} y \cmd{omit}.

\end{itemize}


\section{Instrucciones}
\label{cmd-cmd}

%% auto-generated from XML base, gretl_commands_es.xml
\input{refbody}

\section{Instrucciones arreglaron por tema}
\label{cmd-topics}

%% auto-generated from gretl sources
\input{cmdtopics}




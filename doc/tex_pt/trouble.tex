\chapter{Troubleshooting gretl}
\label{trouble}

\section{Bug reports}
\label{trouble-bugs}

Bug reports are welcome. Hopefully, you are unlikely to find bugs in
the actual calculations done by \app{gretl} (although this statement
does not constitute any sort of warranty). You may, however, come
across bugs or oddities in the behavior of the graphical interface.
Please remember that the usefulness of bug reports is greatly enhanced
if you can be as specific as possible: what \emph{exactly} went wrong,
under what conditions, and on what operating system?  If you saw an
error message, what precisely did it say?

\section{Auxiliary programs}
\label{trouble-programs}

As mentioned above, \app{gretl} calls some other programs to
accomplish certain tasks (gnuplot for graphing, {\LaTeX} for
high-quality typesetting of regression output, GNU R).  If something
goes wrong with such external links, it is not always easy for
\app{gretl} to produce an informative error message.  If such a link
fails when accessed from the \app{gretl} graphical interface, you may
be able to get more information by starting \app{gretl} from the
command prompt rather than via a desktop menu entry or icon.  On the X
window system, start gretl from the shell prompt in an \app{xterm}; on
MS Windows, start the program \cmd{gretlw32.exe} from a console window
or ``DOS box'' using the \verb|-g| or \option{debug} option flag.
Additional error messages may be displayed on the terminal window.

Also please note that for most external calls, \app{gretl} assumes
that the programs in question are available in your ``path'' --- that
is, that they can be invoked simply via the name of the program,
without supplying the program's full location.\footnote{The exception
  to this rule is the invocation of gnuplot under MS Windows, where a
  full path to the program is given.}  Thus if a given program fails,
try the experiment of typing the program name at the command prompt,
as shown below.

\begin{center}
  \begin{tabular}{llll}
    & \textit{Graphing} & \textit{Typesetting} & \textit{GNU R}\\
    X window system & gnuplot & latex, xdvi & R\\
    MS Windows & wgnuplot.exe & pdflatex & RGui.exe\\
  \end{tabular}
\end{center}

If the program fails to start from the prompt, it's not a \app{gretl}
issue but rather that the program's home directory is not in your
path, or the program is not installed (properly).  For details on
modifying your path please see the documentation or online help for
your operating system or shell.
    
%%% Local Variables: 
%%% mode: latex
%%% TeX-master: "gretl-guide"
%%% End: 


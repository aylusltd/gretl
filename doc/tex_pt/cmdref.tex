\chapter{Gretl commands}
\label{cmdref}

\section{Introduction}
\label{cmd-intro}

The commands defined below may be executed interactively in the
command-line client program or in the console window of the GUI
program.  They may also be placed in a ``script'' or batch file for
non-interactive execution.
    
The following notational conventions are used below:
    
\begin{itemize}

\item A \texttt{typewriter font} is used for material that you would
  type directly, and also for internal names of variables.

\item Terms in a \textsl{slanted font} are place-holders: you should
  substitute some specific replacement.  For example, you might type
  \texttt{income} in place of the generic \textsl{xvar}.

\item The construction \texttt{[} \textsl{arg} \texttt{]} means that
  the argument \textsl{arg} is optional: you may supply it or not (but
  in any case don't type the brackets).

\item The phrase ``estimation command'' means a command that generates
  estimates for a given model, for example \cmd{ols}, \cmd{ar} or
  \cmd{wls}.

\end{itemize}

In general, each line of a command script should contain one and only
one complete \app{gretl} command.  There are, however, two means of
continuing a long command from one line of input to another.  First,
if the last non-space character on a line is a backslash, this is
taken as an indication that the command is continued on the following
line.  In addition, if the comma is a valid character in a given
command (for instance, as a separator between function arguments, or
as punctuation in the command \texttt{printf}) then a trailing comma
also indicates continuation.  To emphasize the point: a backslash may
be inserted ``arbitrarily'' to indicate continuation, but a comma
works in this capacity only if it is syntactically valid as part of
the command.

\section{Commands}
\label{cmd-cmd}

%% auto-generated from XML base, gretl_commands.xml
\input{refbody}

\section{Commands by topic}
\label{cmd-topics}

The following sections show the available commands grouped by topic.

%% auto-generated from gretl sources
\input{cmdtopics}



\chapter{Introduzione}
\label{intro}

\section{Caratteristiche principali}
\label{features}

\app{Gretl} � un pacchetto econometrico che comprende una libreria
condivisa, un programma client a riga di comando e un'interfaccia
grafica.
    
\begin{description}
\item[Amichevole] \app{Gretl} offre un'interfaccia utente intuitiva,
  che permette di entrare subito nel vivo dell'analisi econometrica.
  Grazie all'integrazione con i libri di testo di Ramu Ramanathan, di
  Jeffrey Wooldridge, di James Stock e Mark Watson, il pacchetto offre
  molti file di dati e script di comandi, commentati e pronti all'uso.
  Gli utenti di gretl hanno inoltre a disposizione la documentazione
  sul programma e la mailing list \href{http://gretl.sourceforge.net/lists.html}{gretl-users}.
\item[Flessibile] � possibile scegliere il proprio metodo di lavoro
  preferito: dal punta-e-clicca interattivo alla modalit� batch,
  oppure una combinazione dei due approcci.
\item[Multi-piattaforma] La piattaforma di sviluppo di \app{Gretl} �
  Linux, ma il programma � disponibile anche per MS Windows e
  Mac OS X, e dovrebbe funzionare su qualsiasi sistema operativo simile a UNIX
  che comprenda le librerie di base richieste (si veda
  l'appendice~\ref{app-build}).
\item[Open source] L'intero codice sorgente di \app{Gretl} �
  disponibile per chiunque voglia criticarlo, correggerlo o
  estenderlo. Si veda l'appendice~\ref{app-build}.
\item[Sofisticato] \app{Gretl} offre un'ampia variet�
  di stimatori basati sui minimi quadrati, sia per singole equazioni
  che per sistemi di equazioni, compresa l'autoregressione vettoriale e i
  modelli a correzione di errore. Comprende anche vari stimatori
  di massima verosimiglianza (ad es.\ probit, ARIMA, GARCH), mentre altri
  metodi avanzati sono implementabili dall'utente attraverso la stima
  generica di massima verosimiglianza o la stima GMM non lineare.
\item[Estensibile] Gli utenti possono estendere \app{gretl} scrivendo le proprie
  funzioni e procedure in un linguaggio di scripting che include una gamma abbastanza
  ampia di funzioni matriciali.
\item[Accurato] \app{Gretl} � stato testato a fondo con vari dataset di
  riferimento, tra cui quelli del NIST. Si veda l'appendice~\ref{app-accuracy}.
\item[Pronto per internet] \app{Gretl} pu� scaricare i database da un
  server alla Wake Forest University; inoltre, comprende una funzionalit�
  di aggiornamento che controlla se � disponibile una nuova versione del programma
  e, nella versione per MS Windows, permette di aggiornarlo automaticamente.
\item[Internazionale] \app{Gretl} supporta le lingue inglese,
  francese, italiana, spagnola, polacca, portoghese, tedesca, o basca,
  a seconda della lingua impostata sul computer.
\end{description}

\section{Ringraziamenti}
\label{ack}

La base di codice di \app{Gretl} deriva da quella del programma \app{ESL}
(``Econometrics Software Library''), scritto dal Professor
Ramu Ramanathan della University of California, San Diego. Siamo molto
grati al Professor Ramanathan per aver reso disponibile questo codice con
licenza GNU General Public License e per aver aiutato nello sviluppo iniziale
di \app{Gretl}.
 
Siamo anche grati agli autori di molti testi di econometria che hanno
permesso la distribuzione delle versioni \app{Gretl} dei dataset contenuti
nei loro libri. Questa lista al momento comprende William Greene,
autore di \emph{Econometric Analysis}, Jeffrey Wooldridge
(\emph{Introductory Econometrics: A Modern Approach}); James Stock e
Mark Watson (\emph{Introduction to Econometrics}); Damodar Gujarati
(\emph{Basic Econometrics}); Russell Davidson e James MacKinnon
(\emph{Econometric Theory and Methods}); Marno Verbeek (\emph{A
Guide to Modern Econometrics}).

La stima GARCH in \app{Gretl} si basa sul codice pubblicato sul \emph{Journal of
Applied Econometrics} dai Prof. Fiorentini, Calzolari e Panattoni, mentre il
codice per generare i \emph{p}-value per i test Dickey Fuller � di James
MacKinnon. In ognuno dei casi sono grato agli autori per avermi permesso di
usare il loro lavoro.  

Per quanto riguarda l'internazionalizzazione di \app{Gretl}, vorrei ringraziare
Ignacio D�az-Emparanza (spagnolo), Michel Robitaille e Florent Bresson (francese),
Cristian Rigamonti (italiano), Tadeusz Kufel e Pawel Kufel (polacco),
Markus Hahn e Sven Schreiber (tedesco), H�lio Guilherme (Portoghese) e Susan Orbe (Basco).

\app{Gretl} ha beneficiato largamente del lavoro di molti sviluppatori di software
libero e open-source: per i dettagli si veda l'appendice~\ref{app-build}.
Devo ringraziare Richard Stallman della Free Software Foundation per
il suo supporto al software libero in generale, ma in particolare per
aver accettato di ``adottare'' \app{Gretl} come programma GNU.

Molti utenti di \app{Gretl} hanno fornito utili suggerimenti e
segnalazioni di errori. Un ringraziamento particolare a Ignacio
D�az-Emparanza, Tadeusz Kufel, Pawel Kufel, Alan Isaac, Cristian Rigamonti, Sven
Schreiber, Talha Yalta, Andreas Rosenblad e Dirk Eddelbuettel, che cura il pacchetto \app{Gretl}
per Debian GNU/Linux.

\section{Installazione del programma}
\label{install}

\subsection{Linux}
\label{linux-install}

Sulla piattaforma Linux\footnote{In questo manuale
  verr� usata l'abbreviazione ``Linux'' per riferirsi al sistema
  operativo GNU/Linux. Ci� che viene detto a proposito di Linux vale
  anche per altri sistemi simili a UNIX, anche se potrebbero essere
  necessari alcuni adattamenti.}, � possibile compilare da s� il
codice di \app{Gretl}, oppure usare un pacchetto pre-compilato.

Se si vuole utilizzare la versione di sviluppo di \app{gretl}, o modificare il
programma per le proprie esigenze, occorre compilare dai sorgenti; visto che
questa operazione richiede una certa abilit�, la maggior parte degli utenti
preferir� usare un pacchetto pre-compilato.

Alcune distribuzioni Linux contengono gi� \app{gretl}, ad esempio Debian o
Ubuntu (nell'archivio \emph{universe}). Chi usa queste distribuzioni deve solo
installare \app{gretl} usando il proprio gestore di pacchetti preferito (ad es.
\app{synaptic}).

Pacchetti binari in formato \app{rpm} (utilizzabili su sistemi Red Hat Linux e
simili) sono disponibili sul sito di \app{gretl}:
\url{http://gretl.sourceforge.net}.

In ogni caso, speriamo che gli utenti che possiedono conoscenze di
programmazione trovino \app{gretl} abbastanza interessante da migliorare ed
estendere. La documentazione dell'API di libgretl non � ancora completa, ma �
possibile trovare alcuni dettagli seguendo il link
``Libgretl API docs'' sul sito web di \app{gretl}. Chi � interessato allo
sviluppo del programma � invitato a iscriversi alla mailing list
\href{http://gretl.sourceforge.net/lists.html}{gretl-devel}.

Per chi vuole compilare \app{gretl} da s�, le istruzioni si trovano
nell'Appendice~\ref{app-build}.

\subsection{MS Windows}
\label{windows-install}

La versione MS Windows � disponibile sotto forma di file eseguibile
auto-estraente. Per installarlo, occorre scaricare
\verb+gretl_install.exe+ ed eseguire questo programma. Verr� chiesta
una posizione in cui installare il pacchetto.

\subsection{Aggiornamento}
\label{updating}


Se si ha un computer connesso a internet, all'avvio \app{Gretl} pu�
collegarsi al proprio sito web alla Wake Forest University per vedere
se sono disponibili aggiornamenti al programma. In caso positivo,
comparir� una finestra informativa. Per attivare questa funzionalit�,
occorre abilitare la casella ``Avvisa in caso di aggiornamenti di
gretl'' nel men� ``Strumenti, Preferenze, Generali...'' di \app{Gretl}.
      
La versione MS Windows di \app{Gretl} fa un passo in pi�: d� anche la
possibilit� di aggiornare automaticamente il programma. � sufficiente
seguire le indicazioni nella finestra pop-up: chiudere \app{Gretl} ed
eseguire il programma di aggiornamento ``gretl updater'' (che di
solito si trova vicino alla voce \app{Gretl} nel gruppo Programmi del
men� Avvio di Windows). Quando il programma di aggiornamento ha
concluso il suo funzionamento, � possibile avviare di nuovo
\app{Gretl}.
      
%%% Local Variables: 
%%% mode: latex
%%% TeX-master: "gretl-guide-it"
%%% End: 


\chapter{Guida ai comandi}
\label{cmdref}

\section{Introduzione}
\label{cmd-intro}

I comandi descritti in questa guida possono essere eseguiti nella
versione a riga di comando del programma, ma anche in quella con con
interfaccia grafica (GUI), inserendoli in un file ``script'' da
eseguire o nel ``terminale di gretl''.
    
La guida utilizza le seguenti convenzioni:
    
\begin{itemize}
\item Il \texttt{carattere a larghezza fissa} viene usato per indicare
  ci� che l'utente deve scrivere e per i nomi delle variabili.

\item I termini in \textsl{corsivo} sono dei segnaposto da sostituire
  con termini specifici, ad esempio si potrebbe scrivere
  \texttt{reddito} invece del generico \textsl{variabile-x}.

\item Il costrutto \texttt{[} \textsl{arg} \texttt{]} indica che
  l'argomento \textsl{arg} � opzionale: � possibile fornirlo oppure no
  (in ogni caso, le parentesi quadre non vanno scritte).

\item La frase ``comando di stima'' indica un comando che genera stime
  per un certo modello, ad esempio \cmd{ols}, \cmd{ar} o \cmd{wls}.
\end{itemize}

\section{Comandi in ordine alfabetico}
\label{cmd-cmd}

%% auto-generated from XML base, gretl_commands_it.xml
\input{refbody}

\section{Comandi raggruppati per argomento}
\label{cmd-topics}

Di seguito, i comandi disponibili sono raggruppati per argomento.
    
%% auto-generated from gretl sources
\input{cmdtopics}


